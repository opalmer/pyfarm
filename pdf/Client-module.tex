%
% API Documentation for API Documentation
% Module Client
%
% Generated by epydoc 3.0.1
% [Mon Feb  9 15:24:33 2009]
%

%%%%%%%%%%%%%%%%%%%%%%%%%%%%%%%%%%%%%%%%%%%%%%%%%%%%%%%%%%%%%%%%%%%%%%%%%%%
%%                          Module Description                           %%
%%%%%%%%%%%%%%%%%%%%%%%%%%%%%%%%%%%%%%%%%%%%%%%%%%%%%%%%%%%%%%%%%%%%%%%%%%%

    \index{Client \textit{(module)}|(}
\section{Module Client}

    \label{Client}
\begin{alltt}

AUTHOR: Oliver Palmer
CONTACT: oliverpalmer@opalmer.com
INITIAL: Jan 31 2009
PURPOSE: To handle and run all client connections on a remote machine

    This file is part of PyFarm.

    PyFarm is free software: you can redistribute it and/or modify
    it under the terms of the GNU General Public License as published by
    the Free Software Foundation, either version 3 of the License, or
    (at your option) any later version.

    PyFarm is distributed in the hope that it will be useful,
    but WITHOUT ANY WARRANTY; without even the implied warranty of
    MERCHANTABILITY or FITNESS FOR A PARTICULAR PURPOSE.  See the
    GNU General Public License for more details.

    You should have received a copy of the GNU General Public License
    along with PyFarm.  If not, see {\textless}http://www.gnu.org/licenses/{\textgreater}.
\end{alltt}


%%%%%%%%%%%%%%%%%%%%%%%%%%%%%%%%%%%%%%%%%%%%%%%%%%%%%%%%%%%%%%%%%%%%%%%%%%%
%%                               Variables                               %%
%%%%%%%%%%%%%%%%%%%%%%%%%%%%%%%%%%%%%%%%%%%%%%%%%%%%%%%%%%%%%%%%%%%%%%%%%%%

  \subsection{Variables}

    \vspace{-1cm}
\hspace{\varindent}\begin{longtable}{|p{\varnamewidth}|p{\vardescrwidth}|l}
\cline{1-2}
\cline{1-2} \centering \textbf{Name} & \centering \textbf{Description}& \\
\cline{1-2}
\endhead\cline{1-2}\multicolumn{3}{r}{\small\textit{continued on next page}}\\\endfoot\cline{1-2}
\endlastfoot\raggedright C\-F\-G\- & \raggedright \textbf{Value:} 
{\tt os.getcwd()+ '/settings.cfg'}&\\
\cline{1-2}
\raggedright S\-I\-Z\-E\-O\-F\-\_\-U\-I\-N\-T\-1\-6\- & \raggedright \textbf{Value:} 
{\tt Settings.Network(CFG).Unit16Size()}&\\
\cline{1-2}
\raggedright B\-R\-O\-A\-D\-C\-A\-S\-T\-\_\-P\-O\-R\-T\- & \raggedright \textbf{Value:} 
{\tt Settings.Network(CFG).BroadcastPort()}&\\
\cline{1-2}
\raggedright Q\-U\-E\-\_\-P\-O\-R\-T\- & \raggedright \textbf{Value:} 
{\tt Settings.Network(CFG).QuePort()}&\\
\cline{1-2}
\raggedright S\-T\-D\-O\-U\-T\-\_\-P\-O\-R\-T\- & \raggedright \textbf{Value:} 
{\tt Settings.Network(CFG).StdOutPort()}&\\
\cline{1-2}
\raggedright S\-T\-D\-E\-R\-R\-\_\-P\-O\-R\-T\- & \raggedright \textbf{Value:} 
{\tt Settings.Network(CFG).StdErrPort()}&\\
\cline{1-2}
\raggedright U\-S\-E\-\_\-S\-T\-A\-T\-I\-C\-\_\-C\-L\-I\-E\-N\-T\- & \raggedright \textbf{Value:} 
{\tt False}&\\
\cline{1-2}
\raggedright a\-p\-p\- & \raggedright \textbf{Value:} 
{\tt QCoreApplication(sys.argv)}&\\
\cline{1-2}
\raggedright m\-a\-i\-n\- & \raggedright \textbf{Value:} 
{\tt Main()}&\\
\cline{1-2}
\end{longtable}


%%%%%%%%%%%%%%%%%%%%%%%%%%%%%%%%%%%%%%%%%%%%%%%%%%%%%%%%%%%%%%%%%%%%%%%%%%%
%%                           Class Description                           %%
%%%%%%%%%%%%%%%%%%%%%%%%%%%%%%%%%%%%%%%%%%%%%%%%%%%%%%%%%%%%%%%%%%%%%%%%%%%

    \index{Client \textit{(module)}!Client.Main \textit{(class)}|(}
\subsection{Class Main}

    \label{Client:Main}
\begin{tabular}{cccccc}
% Line for ??-1, linespec=[False]
\multicolumn{2}{r}{\settowidth{\BCL}{??-1}\multirow{2}{\BCL}{??-1}}
&&
  \\\cline{3-3}
  &&\multicolumn{1}{c|}{}
&&
  \\
&&\multicolumn{2}{l}{\textbf{Client.Main}}
\end{tabular}


%%%%%%%%%%%%%%%%%%%%%%%%%%%%%%%%%%%%%%%%%%%%%%%%%%%%%%%%%%%%%%%%%%%%%%%%%%%
%%                                Methods                                %%
%%%%%%%%%%%%%%%%%%%%%%%%%%%%%%%%%%%%%%%%%%%%%%%%%%%%%%%%%%%%%%%%%%%%%%%%%%%

  \subsubsection{Methods}

    \label{Client:Main:__init__}
    \index{Client \textit{(module)}!Client.Main \textit{(class)}!Client.Main.\_\_init\_\_ \textit{(method)}}

    \vspace{0.5ex}

\hspace{.8\funcindent}\begin{boxedminipage}{\funcwidth}

    \raggedright \textbf{\_\_init\_\_}(\textit{self}, \textit{parent}={\tt None})

\setlength{\parskip}{2ex}
\setlength{\parskip}{1ex}
    \end{boxedminipage}

    \label{Client:Main:startBroadcast}
    \index{Client \textit{(module)}!Client.Main \textit{(class)}!Client.Main.startBroadcast \textit{(method)}}

    \vspace{0.5ex}

\hspace{.8\funcindent}\begin{boxedminipage}{\funcwidth}

    \raggedright \textbf{startBroadcast}(\textit{self})

\setlength{\parskip}{2ex}
\setlength{\parskip}{1ex}
    \end{boxedminipage}

    \label{Client:Main:initSlave}
    \index{Client \textit{(module)}!Client.Main \textit{(class)}!Client.Main.initSlave \textit{(method)}}

    \vspace{0.5ex}

\hspace{.8\funcindent}\begin{boxedminipage}{\funcwidth}

    \raggedright \textbf{initSlave}(\textit{self})

\setlength{\parskip}{2ex}
\setlength{\parskip}{1ex}
    \end{boxedminipage}

    \index{Client \textit{(module)}!Client.Main \textit{(class)}|)}
    \index{Client \textit{(module)}|)}
